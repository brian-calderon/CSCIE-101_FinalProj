% Options for packages loaded elsewhere
\PassOptionsToPackage{unicode}{hyperref}
\PassOptionsToPackage{hyphens}{url}
\PassOptionsToPackage{dvipsnames,svgnames,x11names}{xcolor}
%
\documentclass[
  letterpaper,
  DIV=11,
  numbers=noendperiod]{scrartcl}

\usepackage{amsmath,amssymb}
\usepackage{iftex}
\ifPDFTeX
  \usepackage[T1]{fontenc}
  \usepackage[utf8]{inputenc}
  \usepackage{textcomp} % provide euro and other symbols
\else % if luatex or xetex
  \usepackage{unicode-math}
  \defaultfontfeatures{Scale=MatchLowercase}
  \defaultfontfeatures[\rmfamily]{Ligatures=TeX,Scale=1}
\fi
\usepackage{lmodern}
\ifPDFTeX\else  
    % xetex/luatex font selection
\fi
% Use upquote if available, for straight quotes in verbatim environments
\IfFileExists{upquote.sty}{\usepackage{upquote}}{}
\IfFileExists{microtype.sty}{% use microtype if available
  \usepackage[]{microtype}
  \UseMicrotypeSet[protrusion]{basicmath} % disable protrusion for tt fonts
}{}
\makeatletter
\@ifundefined{KOMAClassName}{% if non-KOMA class
  \IfFileExists{parskip.sty}{%
    \usepackage{parskip}
  }{% else
    \setlength{\parindent}{0pt}
    \setlength{\parskip}{6pt plus 2pt minus 1pt}}
}{% if KOMA class
  \KOMAoptions{parskip=half}}
\makeatother
\usepackage{xcolor}
\usepackage[margin=1.3cm]{geometry}
\setlength{\emergencystretch}{3em} % prevent overfull lines
\setcounter{secnumdepth}{5}
% Make \paragraph and \subparagraph free-standing
\makeatletter
\ifx\paragraph\undefined\else
  \let\oldparagraph\paragraph
  \renewcommand{\paragraph}{
    \@ifstar
      \xxxParagraphStar
      \xxxParagraphNoStar
  }
  \newcommand{\xxxParagraphStar}[1]{\oldparagraph*{#1}\mbox{}}
  \newcommand{\xxxParagraphNoStar}[1]{\oldparagraph{#1}\mbox{}}
\fi
\ifx\subparagraph\undefined\else
  \let\oldsubparagraph\subparagraph
  \renewcommand{\subparagraph}{
    \@ifstar
      \xxxSubParagraphStar
      \xxxSubParagraphNoStar
  }
  \newcommand{\xxxSubParagraphStar}[1]{\oldsubparagraph*{#1}\mbox{}}
  \newcommand{\xxxSubParagraphNoStar}[1]{\oldsubparagraph{#1}\mbox{}}
\fi
\makeatother


\providecommand{\tightlist}{%
  \setlength{\itemsep}{0pt}\setlength{\parskip}{0pt}}\usepackage{longtable,booktabs,array}
\usepackage{calc} % for calculating minipage widths
% Correct order of tables after \paragraph or \subparagraph
\usepackage{etoolbox}
\makeatletter
\patchcmd\longtable{\par}{\if@noskipsec\mbox{}\fi\par}{}{}
\makeatother
% Allow footnotes in longtable head/foot
\IfFileExists{footnotehyper.sty}{\usepackage{footnotehyper}}{\usepackage{footnote}}
\makesavenoteenv{longtable}
\usepackage{graphicx}
\makeatletter
\def\maxwidth{\ifdim\Gin@nat@width>\linewidth\linewidth\else\Gin@nat@width\fi}
\def\maxheight{\ifdim\Gin@nat@height>\textheight\textheight\else\Gin@nat@height\fi}
\makeatother
% Scale images if necessary, so that they will not overflow the page
% margins by default, and it is still possible to overwrite the defaults
% using explicit options in \includegraphics[width, height, ...]{}
\setkeys{Gin}{width=\maxwidth,height=\maxheight,keepaspectratio}
% Set default figure placement to htbp
\makeatletter
\def\fps@figure{htbp}
\makeatother

\KOMAoption{captions}{tableheading}
\makeatletter
\@ifpackageloaded{caption}{}{\usepackage{caption}}
\AtBeginDocument{%
\ifdefined\contentsname
  \renewcommand*\contentsname{Table of contents}
\else
  \newcommand\contentsname{Table of contents}
\fi
\ifdefined\listfigurename
  \renewcommand*\listfigurename{List of Figures}
\else
  \newcommand\listfigurename{List of Figures}
\fi
\ifdefined\listtablename
  \renewcommand*\listtablename{List of Tables}
\else
  \newcommand\listtablename{List of Tables}
\fi
\ifdefined\figurename
  \renewcommand*\figurename{Figure}
\else
  \newcommand\figurename{Figure}
\fi
\ifdefined\tablename
  \renewcommand*\tablename{Table}
\else
  \newcommand\tablename{Table}
\fi
}
\@ifpackageloaded{float}{}{\usepackage{float}}
\floatstyle{ruled}
\@ifundefined{c@chapter}{\newfloat{codelisting}{h}{lop}}{\newfloat{codelisting}{h}{lop}[chapter]}
\floatname{codelisting}{Listing}
\newcommand*\listoflistings{\listof{codelisting}{List of Listings}}
\makeatother
\makeatletter
\makeatother
\makeatletter
\@ifpackageloaded{caption}{}{\usepackage{caption}}
\@ifpackageloaded{subcaption}{}{\usepackage{subcaption}}
\makeatother

\ifLuaTeX
  \usepackage{selnolig}  % disable illegal ligatures
\fi
\usepackage{bookmark}

\IfFileExists{xurl.sty}{\usepackage{xurl}}{} % add URL line breaks if available
\urlstyle{same} % disable monospaced font for URLs
\hypersetup{
  pdftitle={HARVARD EXTENSION SCHOOL},
  pdfauthor={Author: Dinesh Bedathuru; Author: Brian Calderon; Author: Jeisson Hernandez; Author: Hao Fu; Author: Derek Rush; Author: Jeremy Tajonera; Author: Catherine Tully},
  colorlinks=true,
  linkcolor={blue},
  filecolor={Maroon},
  citecolor={Blue},
  urlcolor={Blue},
  pdfcreator={LaTeX via pandoc}}


\title{HARVARD EXTENSION SCHOOL}
\usepackage{etoolbox}
\makeatletter
\providecommand{\subtitle}[1]{% add subtitle to \maketitle
  \apptocmd{\@title}{\par {\large #1 \par}}{}{}
}
\makeatother
\subtitle{EXT CSCI E-106 Model Data Class Group Project Template}
\author{Author: Dinesh Bedathuru \and Author: Brian
Calderon \and Author: Jeisson Hernandez \and Author: Hao Fu \and Author:
Derek Rush \and Author: Jeremy Tajonera \and Author: Catherine Tully}
\date{06 May 2025}

\begin{document}
\maketitle
\begin{abstract}
In this project, our aim is to classify the probability of a passenger
surviving the Titanic crash of 1912. We used a variety of linear and
non-linear models to deduce the most accurate model and provide
long-term stability in our predictions.
\end{abstract}

\renewcommand*\contentsname{Table of contents}
{
\hypersetup{linkcolor=}
\setcounter{tocdepth}{2}
\tableofcontents
}

\begin{verbatim}
Size of entire data set: 1310 
\end{verbatim}

Classify whether a passenger on board the maiden voyage of the RMS
Titanic in 1912 survived given their age, sex and class.
Sample-Data-Titanic-Survival.csv to be used in the Final Project

\begin{longtable}[]{@{}
  >{\raggedright\arraybackslash}p{(\columnwidth - 2\tabcolsep) * \real{0.5000}}
  >{\raggedright\arraybackslash}p{(\columnwidth - 2\tabcolsep) * \real{0.5000}}@{}}
\toprule\noalign{}
\begin{minipage}[b]{\linewidth}\raggedright
Variable
\end{minipage} & \begin{minipage}[b]{\linewidth}\raggedright
Description
\end{minipage} \\
\midrule\noalign{}
\endhead
\bottomrule\noalign{}
\endlastfoot
pclass & \textbf{Passenger Class (1 = 1st; 2 = 2nd; 3 = 3rd)} \\
survived & \textbf{Survival (0 = No; 1 = Yes)} \\
name & \textbf{Name} \\
sex & \textbf{Sex} \\
age & \textbf{Age} \\
sibsp & \textbf{\# of siblings / spouses aboard the Titanic} \\
parch & \textbf{\# of parents / children aboard the Titanic} \\
ticket & \textbf{Ticket number} \\
fare & \textbf{Passenger fare} \\
cabin & \textbf{Cabin number} \\
embarked & \textbf{Port of Embarkation (C = Cherbourg; Q = Queenstown; S
= Southampton)} \\
boat & \textbf{Lifeboat ID, if passenger survived} \\
body & \textbf{Body number (if passenger did not survive and body was
recovered} \\
home.dest & \textbf{The intended home destination of the passenger} \\
\end{longtable}

\section{Instructions:}\label{instructions}

0. Join a team with your fellow students with appropriate size (Up to
Nine Students total) If you have not group by the end of the week of
April 11 you may present the project by yourself or I will randomly
assign other stranded student to your group. I will let know the final
groups in April 11.

1. Load and Review the dataset named ``Titanic\_Survival\_Data.csv'' 2.
Create the train data set which contains 70\% of the data and use
set.seed (15). The remaining 30\% will be your test data set.

3. Investigate the data and combine the level of categorical variables
if needed and drop variables as needed. For example, you can drop id,
Latitude, Longitude, etc.

4. Build appropriate model to predict the probability of survival.

5. Create scatter plots and a correlation matrix for the train data set.
Interpret the possible relationship between the response.

6. Build the best models by using the appropriate selection method.
Compare the performance of the best logistic linear models.

7. Make sure that model assumption(s) are checked for the final model.
Apply remedy measures (transformation, etc.) that helps satisfy the
assumptions.

8. Investigate unequal variances and multicollinearity.

9. Build an alternative to your model based on one of the following
approaches as applicable to predict the probability of survival:
logistic regression, classification Tree, NN, or SVM. Check the
applicable model assumptions. Explore using a negative binomial
regression and a Poisson regression.

10. Use the test data set to assess the model performances from above.

11. Based on the performances on both train and test data sets,
determine your primary (champion) model and the other model which would
be your benchmark model.

12. Create a model development document that describes the model
following this template, input the name of the authors, Harvard IDs, the
name of the Group, all of your code and calculations, etc..

\textbf{Due Date: May 12 2025 1159 pm hours EST Notes No typographical
errors, grammar mistakes, or misspelled words, use English language All
tables need to be numbered and describe their content in the body of the
document All figures/graphs need to be numbered and describe their
content All results must be accurate and clearly explained for a casual
reviewer to fully understand their purpose and impact Submit both the
RMD markdown file and PDF with the sections with appropriate
explanations. A more formal.}

\section{Executive Summary}\label{executive-summary}

This section will describe the model usage, your conclusions and any
regulatory and internal requirements. In a real world scneario, this
section is for senior management who do not need to know the details.
They need to know high level (the purpose of the model, limitations of
the model and any issues).

\section{I. Introduction (5 points)}\label{i.-introduction-5-points}

This section needs to introduce the reader to the problem to be
resolved, the purpose, and the scope of the statistical testing applied.
What you are doing with your prediction? What is the purpose of the
model? What methods were trained on the data, how large is the test
sample, and how did you build the model?

The Titanic was a British-registered ship that set sail on its maiden
voyage on April 10th, 1912 with 2,240 passengers and crew on board. On
April 15th, 1912, the ship struck an iceberg, split in half, and sank to
the bottom of the ocean (REF 1). In this report, we are going to analyze
the data in the Titanic.csv file and use it to determine the best model
for predicting whether someone on board would live or die. By creating
this model, we hope to understand what factors a passenger could have
taken into account in order to reduce their risk of death during the
trip. We cleaned the data and split into into a train/test split in
order to properly train our models. We created simple linear models,
multivariate linear models, logistic models (both binomial and poisson),
a regression tree, and a neural network model. The train sample size was
916 data points (70.03058\%) and the test sample size was 392 data
points (29.96942\%). We built the models after examining the data and
determining which predictor variables we thought would be most relevant
for survival rate. Once we had our variables and training data, we
created the models and examined the performance of the models on both
training and testing data to determine if they were robust. We also
examined if the model assumptions appeared to hold for each model.

\begin{verbatim}
[1] 4
\end{verbatim}

The \texttt{echo:\ false} option disables the printing of code (only
output is displayed).




\end{document}
